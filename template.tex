%%% Homework template
% Derived from https://www.overleaf.com/articles/math-453-hw-1/kxfhqwwnygkk

% https://math.mit.edu/~psh/exam/examdoc.pdf
\documentclass[addpoints]{exam}

%%% Packages
\usepackage{amsmath,enumitem,wrapfig}
\usepackage{tikz}
\usepackage{fancyhdr}
\usepackage{lastpage}
\usepackage{color}

%%% Fonts
% https://wiki.carleton.edu/download/attachments/20155418/fontguide.pdf
\usepackage{newpxtext,newpxmath}
% \usepackage[charter]{mathdesign}

%%% Commands
\newcommand{\course}{Analysis II}
\newcommand{\name}{Alex Han}
\newcommand{\pset}{Problem Set 1}
\newcommand{\duedate}{Feb 1}

%%% Header and footer
\pagestyle{headandfoot}
\runningheadrule
\firstpageheadrule
\firstpageheader{\course}{\name}{\pset}
\runningheader{\course}{\name}{\pset}
\firstpagefootrule
\runningfootrule
% \firstpagefooter{}{\thepage}{}
% \runningfooter{}{\thepage}{}
\firstpagefooter{}{\thepage\ of \pageref{LastPage}}{}
\runningfooter{}{\thepage\ of \pageref{LastPage}}{}

%%% Solutions format
\printanswers
\shadedsolutions
% http://latexcolor.com/
\definecolor{SolutionColor}{rgb}{0.96, 0.96, 0.96}
% \definecolor{SolutionColor}{rgb}{0.94, 0.97, 1.0}
% \unframedsolutions
% \renewenvironment{solution}{\\\\\textbf{Solution:}\\\\}{\\}

\begin{document}

\begin{questions}

	\question \textbf{M1}
	\begin{parts}
		\part
		Consider the arithmetic computation below.
		\begin{align}
			3+4[5-12]-6(3) +(4+0)
			 & = 3+4[5-12]-6(3) +4 \\
			 & = 4[5-12]-6(3) +4+3 \\
			 & = 20-48-18 +4+3     \\
			 & =-39.\notag
		\end{align}
		For each of the steps (1), (2), and (3) identify which of the Axioms of Integer Arithmetic are used in the simplification step.

		\begin{solution}
			$3+4[5-12]-6(3) +4$....(1) additive identity\\
			$4[5-12]-6(3) +4+3$ ....(2) commutativity of addition\\
			$20-48-18 +4+3$ ....(3) distributive
		\end{solution}

		\part Create and simplify an expression that uses associativity of addition, multiplicative identity, and the distributive law.

		\begin{solution}\\
			(1) Associativity of addition\\
			=a + (b + c) = (a + b) + c\\
			(2) Multiplication identity\\
			1*a =a\\
			(3) Distributive law\\
			a (b + c) = ab + ac\\
			Example) \\
			$3 + 4(6 + 4) + (7(3) + 5(1))$\\
			= $(3 + 4(6 + 4)) + 7(3) + 5(1)$ ...used (1)\\
			= $(3 + (24 + 14)) + 7(3) + 5(1)$ ....used (3)\\
			= $(3 + 24 + 14) + 21 + 5$ .... used (2)\\
			= $67$
		\end{solution}

	\end{parts}

	\newpage

	\question \textbf{M2}
	For each statement below determine whether each statement is correct for integers $a$, $b$, and $c$. If the statement is correct, then prove it. If the statement is incorrect, then modify it so that it is correct. Be sure to state which Order Axiom(s) you have applied.
	\begin{parts}
		\part If $a<b$, then $c\cdot a< c\cdot b$.
		\begin{solution}
			incorrect.\\
			If $a<b$, then $c\cdot a< c\cdot b$ when $c>0$
		\end{solution}

		\part If $a<b$, then $a+c<b+c$.
		\begin{solution}
			Correct\\
			If $a<b$, then we can assume that $a + r= b$, where r is in Z\\
			Hence, $a + c + r = b + c$, when c is in Z\\
			Therefore, then $a+c<b+c$.

			If $a<b$, then we can assume that $a + r= b$, where r is in Z\\
			Hence, $a + c + r = b + c$, when c is in Z\\
			Therefore, then $a+c<b+c$.

			If $a<b$, then we can assume that $a + r= b$, where r is in Z\\
			Hence, $a + c + r = b + c$, when c is in Z\\
			Therefore, then $a+c<b+c$.
		\end{solution}

		\part If $a<b$, $b<c$, and $c<d$, then $a<d$.

		\begin{solution}
			with the same way part (b)
			Since If $a<b$, then If $a+r=b$, when r is in Z\\
			and  $b<c$, so it follows that $b+r=c$, it is also same with $a+r+r=b+r=c$\\
			also, if $c<d$, then $c+r=d$, it is same with $a+r+r+r=b+r+r=c+r=d$\\
			Therefore, $a+3r=d$\\
			Hence, $a<d$

		\end{solution}
		\part If $a\not > b$ and $a\not< b$, then $a=b$.

		\begin{solution}
			Given that $a\not > b$ and $a\not< b$\\
			For $a\not > b$ \\
			then, it can be either $a < b$ or $a=b$, but $a < b$ is a contradiction by given $a\not< b$.\\
			For $a\not< b$,\\
			then,  it can be either $a > b$ or $a=b$, but $a > b$ is a contradiction by given $a\not> b$.\\
			Therefore, $a=b$.
		\end{solution}
	\end{parts}


\end{questions}
\end{document}
