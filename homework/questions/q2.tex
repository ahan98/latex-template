\begin{question}
  (Chapter 3, Problem 1.1) Let \( \mathbf{X} \) be distributed with density \( f(\mathbf{x} - \xi) = f(x_1-\xi, \ldots, x_n-\xi) \), and let \( \delta \) be equivariant for estimating \( \xi \) with loss function \( L(\xi, d) = \rho(d - \xi) \). Prove that the (a) risk and (b) variance are constant (i.e., do not depend on \( \xi \)).
\end{question}

\begin{solution}
  As in the proof from Theorem 1.4~\cite[p.~150]{tpe} that bias is constant, we use the fact that, if \( \mathbf{X} \) has density \( f(\mathbf{x} - \xi) \), then \( \mathbf{X} + \xi \) has density \( f(\mathbf{x}) \). That is, \( E_\xi[\delta(\mathbf{X})] = E_0[\delta(\mathbf{X} + \xi)]\).

  To show that risk (i.e., expected loss) is constant,
  \[
    \begin{aligned}
      E_\xi[L(\xi, \delta(\mathbf{X}))]
       & = E_0[L(\xi, \delta(\mathbf{X} + \xi))] \\
       & = E_0[L(\xi, \delta(\mathbf{X}) + \xi)] \\
       & = E_0[L(0, \delta(\mathbf{X}))],
    \end{aligned}
  \]
  where the second and third equalities follow from the equivariance of \( \delta \) and \( L \), respectively.

  To show that variance is constant,
  \[
    \begin{aligned}
      \mathrm{var}_\xi[\delta(\mathbf{X})]
       & = \mathrm{var}_0[\delta(\mathbf{X} + \xi)] \\
       & = \mathrm{var}_0[\delta(\mathbf{X}) + \xi] \\
       & = \mathrm{var}_0[\delta(\mathbf{X})],
    \end{aligned}
  \]
  where the second equality follows from the equivariance of \( \delta \).
\end{solution}