\chapter{Lecture 03-29}

\begin[Markov Inequality]{theorem}
  \label{thm:markov-inequality}
  Let \( f: \mathbb{ R } \rightarrow [-\infty, \infty] \) be a measurable function. Then, for any \( \lambda \in (0, \infty) \),
  \[
    m \left\{ \abs{f} \ge \lambda \right\} \le  \frac{1}{\lambda} \int \abs{f}
  \]
\end{theorem}

\begin{proof}
  TODO
\end{proof}

\begin{corollary}
  If \( f: \mathbb{ R } \rightarrow [-\infty, \infty] \) is integrable, then \( f \)  is finite a.e.
\end{corollary}

\begin{proof}
  \( A_{\infty} \coloneqq \left\{ \abs{f} = \infty \right\} \). WTS 
\end{proof}

\begin{corollary}
  \( f_{n} \ge  0, f_n \rightarrow f \) a.e., \( \int_{\mathbb{ R }} f_n \rightarrow \int_{\mathbb{ R }} f\) as \( n \rightarrow \infty. \) Then \( \forall  \) measurable subset \( X \subset \mathbb{ R } \) , \( \lim_{n \to \infty} \int_X f_n = \int f_x f = \int f 1_X\) 
\end{corollary}

Tonelli theorem

Fatou

Borel-Cantelli (can be proved by Tonelli)

Relationship b.t. Lebesgue & Riemann integral
- Riemann integrable implies Lebesgue integrable
- Riemann integrable iff continuous a.e.

Example: Characteristic function of bounded measurable subset is Riemann integrable