\chapter{Definitions, Problems and Theorems}
\section{Definitions}
We can define a term like \emph{hamster}\define{hamster}, or say that the term hamster\SubIndex{hamster} appears again later.
\begin{Verbatim}[frame=leftline]
  \chapter{Definitions, Problems and /s}
  \section{Definitions}
  We can define a term like \emph{hamster}\define{hamster},
  or say that the term hamster\SubIndex{hamster} appears again later.
\end{Verbatim}
Compile, for a book called \verb!filename.tex!, with
\begin{Verbatim}[frame=leftline]
  makeindex filename
\end{Verbatim}
We add notation like when we use a variable called \(\omega\),
we put it in the list of notation.%
\Notation{omega}{\omega}{A variable called $\omega$}
\begin{Verbatim}
  We add notation like when we use a variable called \(\omega\),
  we put it in the list of notation.%
  \Notation{omega}{\omega}{A variable called $\omega$}
\end{Verbatim}
If you use notation, compile with
\begin{Verbatim}[frame=leftline]
  makeindex -s notation.gst -o not.gls not.glo
\end{Verbatim}
\section{Problems}
\begin{problem}{label.for.the.first.problem}
What is the point of your life?
\end{problem}
\begin{answer}{label.for.the.first.problem}
  Your life is pointless.
\end{answer}
In problem~\ref{problem:label.for.the.first.problem}, we can clearly see ...
\begin{Verbatim}[frame=leftline]
  % We add problems by:
  \begin{problem}{label.for.the.first.problem}
  What is the point of your life?
  \end{problem}
  % and answers by:
  \begin{answer}{label.for.the.first.problem}
    Your life is pointless.
  \end{answer}
  In problem~\ref{problem:label.for.the.first.problem}, we can clearly see ...
\end{Verbatim}
