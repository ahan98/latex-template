\lesson{9}{di 10 dec 2019 10:31}{}

Recap: 

\[
    0 \to H_1(S^1) = \Z \xrightarrow{inj} H_0(U \cap V)
.\] 
and $\left<c + d \right> \mapsto x-y$.

\begin{figure}[H]
    \centering
    \incfig{homology-of-circle}
    \caption{homology of circle}
    \label{fig:homology-of-circle}
\end{figure}

For $S^{n}$.
Then we had
\[
    0 \to  H_n(S^{n}) = \Z \xrightarrow{\Delta, \cong}  H_{n-1}(U \cap  V) \xrightarrow{r_*} (S^{n-1}) = \Z
.\] 


\begin{corollary}
    There is no retraction of $D^{n} = B^{n} = $ $n$-dimensional disc to $S^{n-1}$.
\end{corollary}
\begin{proof}
    Suppose there is a retract.
    Then
    \[
        \begin{tikzcd}
            S^{n-1} \arrow[dr, "i"]  \arrow[r, "1_{S^{n-1}}"] &S^{n+1}\\
                                    & D^{n} \arrow[u, "r"]
        \end{tikzcd}
    \]

    \[
        \begin{tikzcd}
            \Z = H_{n-1}(S^{n-1}) \arrow[dr, "i*"] \arrow[r, "(1_{S^{n}})_* = 1_{H_{n-1}}"] & H_{n-1}(S^{n-1}) = \Z\\
                                           & H_{n-1}(D^{n}) = 0 \arrow[u, "r_*"]
        \end{tikzcd}
    \]
\end{proof}
\begin{corollary}[Brouwers fixed point theorem]
    Let $n\ge 1$.
    Then any continuous map $f: D^{n} \to D^{n}$ has a fixed point. (Note: this time, for any dimension!)
\end{corollary}

\section*{Maps from spheres to spheres}

Let $f: S^{n} \to  S^{n}$.
Then $f$ induces a map $f_*: H_n(S^{n}) \to  H_n(S^{n})$, so $f_*: \Z \to  \Z$,
or we can write $f_*: \left< \alpha\right> \mapsto m \left<\alpha \right>$, where $\alpha$ is the generator.
\begin{definition}
    The degree of $f$ is $m$. 
    We write $d(f) = m$.
\end{definition}
\begin{prop} Few properties of the degree:
    \begin{enumerate}
        \item $d(\text{Id}) = 1$
        \item  $d(f  \circ  g) = d(f) d(g)$ 
        \item  $f$ constant, or if $f$ is not surjective, then $d(f) = 0$.
        \item If $f \simeq g$, then $d(f) = d(g)$, because they induce the same map (Problem 8)
            (Deep theorem:  $f \simeq g$ iff $d(f) = d(g)$.)
        \item If $f$ is a homotopy equivalence, then $d(f) = \pm 1$.
            (We have a homotopy inverse, and  $f_*  \circ  g_* = 1$?, so the only possibilities are both $1$ or $-1$.
    \end{enumerate}
\end{prop}
\begin{proof}
    We prove 3.
    Suppose $p \not\in  \im f$.
    Then
    \[
        \begin{tikzcd}
            S^{n} \arrow[dr, "f'"] \arrow[r, "f"] & S^{n}\\
                                                 & S^{n} \setminus \{p\}  \arrow[u, "i"]
        \end{tikzcd}
    \]
    Then
    \[
        \begin{tikzcd}
            H_n(S^{n}) \arrow[dr, "f'_*"] \arrow[r, "f_*"] & H_n(S^{n})\\
                                                           & H_n(S^{n} \setminus \{p\}) = 0  \arrow[u, "i_*"]
        \end{tikzcd}
    \]
    where $f'$ is $f$ restricted to the image.
    We see that $f_* = 0$.
\end{proof}



\begin{theorem}
    Let $S^{n} = \{(x_1, \ldots, x_{n+1})\in \R^n  \mid  \sum x_i^2 = 1\}$ and $f: S^{n} \to  S^{n}$ 
    \[
        f: (x_1, x_2, \ldots, x_{n+1}) \mapsto 
        (-x_1, x_2, \ldots, x_{n+1})
    .\] 
    Then $d(f) = -1$.
\end{theorem}

\begin{proof}
    By induction on $n$.

    Base case, $n=1$.
    $S^1 = \{(x_1, x_2)  \mid  x_1^2 + x_2^2 = 1\}$.

\begin{figure}[H]
    \centering
    \incfig{base-case-of-theorem}
    \caption{base case of theorem}
    \label{fig:base-case-of-theorem}
\end{figure}

Note that $f(x) = y$,  $f(y) = x$.
Denote  $U = S \setminus \{z'\} $, $V = S \setminus \{z\} $.
And $f\left(U \right) \subset U $ and $f(V) \subset V$.

Then

\[
    \begin{tikzcd}
        0 \arrow[r, ""] & H_1(S) \arrow[r, "\Delta"] \arrow[d, "f_*"]& H_0(U \cap V) \arrow[d, "(f|_{U \cap V})_*"]\\
        0 \arrow[r, ""] & H_1(S) \arrow[r, "\Delta"] & H_0(U \cap V)
    \end{tikzcd}
\]
TODO: right arrows tails

Then
\[
    (f|_{U \cap V})_* (x-y) = f(x) - f(y) = y - x =  - (x-y)
.\] 
So $\Delta$ is injective, we have that  $f_*(\left<\alpha \right>) = - \left<\alpha \right>$, so $d(f) = -1$.


\hr

Now, assume that the theorem holds for $S^{n-1}$.

Define $U$, $V$ in a similar way.
$f(U) \subset U$ and $f(V) \subset V$.
Again, we have a commutative diagram.

% \[
%     \begin{tikzcd}
%         H_n(S^{n}) \arrow[r, "\Delta, \cong"] & H_{n-1}(U \cap V) \arrow[r, "r_*"] & H_{n-1}(S^{n-1}) \arrow[l, "i_*"]\\
%         H_n(S^{n}) \arrow[r, "\Delta, \cong"] & H_{n-1}(U \cap V) \arrow[r, "r_*"] & H_{n-1}(S^{n-1}) \arrow[l, "i_*"]
%     \end{tikzcd}
% \]

Question: is it true that $f_*(\left<\alpha \right>)$ is just $- \left<\alpha \right>$?

Now, we know that the $f_*$ on the right is simply $\cdot  (- 1)$.
Then, going the whole way around,
\[
    f_*(\left<\alpha \right>) = \Delta ^{-1} (i_*(-1 (r_* (\Delta(\alpha))))
.\] 
This is 
\[
    - \Delta^{-1} ( i_* r_* \Delta (\left<\alpha \right>))
.\] 
As $i_*$ and $r_*$ are inverses, we have that $d(f) = -1$.
\end{proof}



Suppose $f: S^{n} \to  S^{n}$.
Then there is a map $\Sigma f: S^{n+1} \to  S^{n+1}$ (the suspension of $f$), given by

\begin{figure}[H]
    \centering
    \incfig{definition-of-suspension}
    \caption{definition of suspension}
    \label{fig:definition-of-suspension}
\end{figure}

Using formulas: 
\[
    S^{n+1} = \{(x, t) : x \in \R^{n+1}, t \in \R, \|x\|^2 + t^2 = 1\} 
.\] 
Then 
\begin{align*}
    \Sigma f: S^{n+1} &\longrightarrow S^{n+1} \\
    (x, t) &\longmapsto \begin{cases}
        (x, t) & \text{if $x = 0$, north and south pole}\\
        \left(\|x\| f(\frac{x}{\|x\|}), t\right) & \text{if $x \neq 0$}
    \end{cases}
.\end{align*}

Not too difficult to prove that $\Sigma f$ is continuous.
Claim:  $d(f) = d(\Sigma f)$.
\begin{proof}
    Exactly the same technique as we used to prove the previous theorem.
\end{proof}

\begin{remark}
    Suspension is a general technique.
    Let $X$ be a topological space.
    Then $\Sigma X$, the suspension of $X$ is defined as follows:

\begin{figure}[H]
    \centering
    \incfig{definition-of-suspension-general}
    \caption{definition of suspension general}
    \label{fig:definition-of-suspension-general}
\end{figure}

$\Sigma X = X \times [-1, 1] / \sim $ with
\[
    (x, t) \sim  (x', t') \iff t = t' = 1 \text{ or } t = t' = -1
.\] 

Check for yourself: $\Sigma S^{n} = S^{n+1}$ (pointy sphere).
\end{remark}

\begin{corollary}
    If $f_{i}(x_1, \ldots, x_{n+1}) = (x_1, \ldots, -x_i, \ldots, x_{n+1})$.
    Then $d(f) = -1$.
\end{corollary}
\begin{proof}
    Let $h: S^{n} \to  S^{n}$ be the map that exchanges the first coordinate and the $i$th coordinate.
    Then $h$ is a homeomorphism: $h^{-1} = h$, so $h  \circ  h = 1_{S^{n}}$, so $d(h) = \pm 1$
    Then $ f_i = h  \circ  f  \circ  h .$ 
    So
    \[
        d(f_i) = d(h)^2 d(f) = (\pm 1)^2 (-1) = -1
    .\] 
\end{proof}

There is a big difference between even and odd dimensional spheres:

\begin{corollary}
    Let $A: S^{n} \to  S^{n}$ defined by \[
        A(x_1, \ldots,  x_{n+1}) = (-x_1, \ldots, -x_{n+1}).\footnote{Note the error in the book. $A$ should have $n+1$ inputs. TODO}
\]
    Then $d(A) = (-1)^{n+1}$.
\end{corollary}
\begin{proof}
    \[
        d(A) = d(f_1) d(f_2) \cdots  d(f_{n+1}) = (-1)^{n+1}
    .\] 
\end{proof}

So for even dimensional spheres, $d(A) = -1$ and for odd dimensional sphere,  $d(A) = 1$.
This is a \emph{big} difference.

\begin{theorem}
    Let $f: S^{n} \to  S^{n}$ and $g: S^{n} \to  S^{n}$.
    When $f(x) \neq g(x)$ for all $x \in S^{n}$.
    Then $f \simeq A  \circ g$.
\end{theorem}
\begin{proof}
\begin{figure}[H]
    \centering
    \incfig{proof-prop-123}
    \caption{proof prop 123}
    \label{fig:proof-prop-123}
\end{figure}

The line does not go through the center, as $f(x) \neq g(x)$.
Define a homotopy, 
\[
    (x, t) \mapsto  \frac{(1-t) f(x) + t A g(x)}{\|(1-t) f(x) + t Ag(x)\| \neq 0}
.\] 
This is a homotopy between $f$ and $Ag$
\end{proof}

Not in the text:
\begin{corollary}
    If $|d(f)| \neq |d(g)|$, then there is a point $x$ such that $f(x) = g(x)$.
\end{corollary}
\begin{proof}
    The previous theorem says that $d(f) = d(A) d(g)$, so  $|d(f)| = |d(g)|$. 
    You can even make this better by making a distinction between odd and even dimensional spheres.
\end{proof}
\begin{corollary}
    If $|d(f)| \neq 1$, then $f$ has a fixed point, i.e.\ an $x$ such that $f(x) = x = \text{Id}(x)$.
\end{corollary}
\begin{proof}
    If $|d(f)| \neq d(\text{Id}) = 1$, then \ldots
\end{proof}

\begin{theorem}
    Let $f: S^{2n} \to S^{2n}$.
    Then there exists an $x$ in  $S^{2n}$ such that $f(x) = x$ or  $f(x) = -x$.
\end{theorem}
\begin{proof}
    Suppose there is no $x$ such that $f(x) = x$ and $f(x) = -x$.

    \begin{itemize}
        \item Since  $f(x) \neq x$ for all  $x \in X$, we have that $f \simeq A$.
        \item Since  $f(x) \neq -x = A(x)$ for all  $x \in X$, we have that $f \simeq A  \circ A = 1$.
    \end{itemize}
   From this it follows that
   \[
       d(f) = (-1)^{2n + 1} = -1 \qquad 
    d(f) = 1
   ,\] 
   which is a contradiction.
\end{proof}
\begin{remark}
    This does not hold for odd dimensional spheres.
    For example, consider $S_1$, and make a small rotation, then no point is mapped onto itself or onto its antipodal point.
    Doing the same for $S_2$, we see that when we do a small rotation, the north and south pole are switched.
\end{remark}
\begin{remark}
    It's harder to show that for odd dimensional spheres, there always exist such a map.
\end{remark}

Now, we prove the hairy ball theorem, which only holds in even dimensional spheres.

\begin{corollary}
    There is no continuous map $f: S^{2n}\to S^{2n}$ such that $x$ and $f(x)$ are orthogonal for all $x$.
\end{corollary}
\begin{proof}
    By the previous theorem, there is a point for which $f(x) = \pm x$, and  $x \not \perp \pm x$.
\end{proof}
\begin{corollary}
    There exists no nonzero vector field $\phi$ on $S^{2n}$.
\end{corollary}
\begin{proof}
    Suppose $\phi$ does exists.
    Then, map the tangent vector to a point on the sphere:
    \[
        \psi: S^{2n} \to  S^{2n} : x \mapsto \frac{\phi(x)}{\|\phi(x)\|}
    .\] 
\end{proof}

\begin{figure}[H]
    \centering
    \incfig{hairy-ball-theorem}
    \caption{hairy ball theorem}
    \label{fig:hairy-ball-theorem}
\end{figure}
\begin{remark}
    For odd dimensional spheres, there always exists a such a vector field.
    We even know how many linearly independent vector fields exist!
\end{remark}

\hr


\subsection*{Torus}
Goal: $H_n(T^2)$.

\begin{figure}[H]
    \centering
    \incfig{homology-of-torus}
    \caption{homology of torus}
    \label{fig:homology-of-torus}
\end{figure}

 and , and $U \cap V = B^2 \setminus \{ p\} $.

\begin{itemize}
    \item Choose $U = T^2 \setminus \{p\} $
        Then $U$ retracts to the figure eight space (boundary of square)
        Last week, we saw that $H_0(U) = \Z$ and $H_1(U) = \Z \oplus \Z$, with generators $a$ and $b$, and $H_n(U) = 0$ for $n \ge 2$.
    \item  $V = B^2$. As $B^2$ is path connected, $H_0(U) = \Z $ and $H_n(U) = 0$ for all  $n \ge  1$
    \item $U \cap  V = B^2 \setminus \{ p\} $, which retracts to a circle, so $H_0(U \cap V) = H_1(U \cap  V) = \Z$.
\end{itemize}

\centerline{$
            H_2(U) \oplus H_2(U) \to  H_2(T^2) \to  H_1(U \cap V) \to  H_1(U) \oplus H_1(U) \to  H_1(T^2) \to  H_0(U \cap  V) \to  H_0(U) \oplus H_0(U) \to  H_0(T^2)
        $}
    So we have
    \[
        \to  0 \oplus 0 \to  H_2(T^2) \to  \Z \to  \Z^2 \oplus 0 \to  H_1(T^2) \to  \Z \to  \Z \oplus \Z \to  \Z \to  0
    .\] 

    Now, $\Z \to \Z \oplus \Z$ is injective, as the kernel of $\Z\oplus \Z \to  \Z$ is non-trivial. ($1$ gets mapped to something non-trivial)
    So we can shorten the sequence

    \[
        \to  0 \oplus 0 \to  H_2(T^2) \to  \Z \xrightarrow{g_*}   \Z^2 \oplus 0 \to  H_1(T^2) \to  0
    .\] 
    Now, we want to know how $g_*$ works.
    We need a generator of $\Z$ and look at its embedding?
    Then $\Z = \left<c_1 + c_2 + c_3 + c_4 \right>$.
    Then

    \[
        \begin{tikzcd}
            U \cap V \arrow[r, "i"] & U \arrow[r, "r"] & \text{figure 8}\\
            H_1(U \cap V) \arrow[r, "i_r"] & H_1(U) \arrow[r, "r_*"] & H_1(\text{figure 8})\\
            \left<c_1 + c_2 + c_3 + c_4 \right> \arrow[r, ""] & \left<c_1+ c_2 +c_3 +c_4 \right> \arrow[r, ""] & \left<r  \circ  c_1 + r  \circ  c_2 + r  \circ  c_3 + r  \circ  c_4 \right>\\
                                  && \left< \overline{a} + \overline{b} + a + b \right>
        \end{tikzcd}
    \]
    Now, as these are all cycles, we have that this is 
    \[
    \left<\overline{a} \right> + \left<\overline{b} \right> + \left<a \right> + \left<b \right>
    .\] 
    Now, using Problem 6, we find that this is
    \[
    - \left<a \right> - \left<b \right> + \left<a \right> + \left<b \right> = 0
    .\] 
    Therefore, $g_*$ is the zero map.
    \[
        \to  0 \oplus 0 \to  H_2(T^2) \to  \Z \xrightarrow{0}   \Z^2 \oplus 0 \to  H_1(T^2) \to  0
    ,\] 
    so the kernel of $\Z^2 \oplus 0 \to  H_1(T^2)$ is an injective map.
    As $H_1(T^2) \to  0$, this implies that $\Z^2 \oplus 0 \to  H_1(T^2)$ is a surjective map.
    Therefore, $H_1(T^2) = \Z^2$.
    Now, $H_2(T^2) \to H_1(U \cap V)$ is surjective, and it is also injective, so $H_2(T^2) = \Z$.
    

    % (TODO: trick: if zero, then add > to arrow left or right same if a map is zero)

Now, for $n \ge  3$.
Then
\[
    H_n(U) \oplus H_n(V) \to  H_n(T^2) \to  H_{n-1}(U \cap V).
.\] 
This simplifies  to
\[
    0 \oplus 0 \to  H_n(T^2) \to 0.
,\] 
so $H_n(T^2) = 0$.
Conclusion:
\[
    H_0(T^2) = \Z \quad 
    H_1(T^2) = \Z^2 \quad 
    H_2(T^2) = \Z \quad 
    H_n(T^2) = 0 \text{ for all $ n > =3$}
.\] 
\begin{remark}
    $H_2(S^1 \times S^1) \neq H_2(S^{1}) \times H_2(S^{1})$.
\end{remark}


\begin{remark}
    $H_n(T^{k}) = \Z^{n \choose k}$, where ${n \choose k} = 0$ when  $k > n$.
\end{remark}




\subsection*{Klein bottle}

\begin{figure}[H]
    \centering
    \incfig{klein-bottle-homology}
    \caption{klein bottle homology}
    \label{fig:klein-bottle-homology}
\end{figure}

\begin{itemize}
    \item $U = K \setminus \{p\}$, which retracts to the figure eight space (boundary of the square)
        So $H_0 = \Z$ and $H_1  = Z^2 = \operatorname{grp}(\left<a \right>, \left<b \right>)$, and $H_n =0$ for  $n \ge  2$.
    \item $V = B^2$
        So $H_0 = \Z$ and $H_n = 0$ for  $n \ge 1 $.
    \item $U \cap V = B^2 \setminus \{p\}$, which retracts to a circle.
        So $H_0 = \Z$ and $H_1 = \Z = \operatorname{grp}(\left<c_1 + c_2 + c_3 + c_4\right>$)
\end{itemize}

\centerline{$
            \to  H_2(U) \oplus H_2(V) \to  H_2(K) \to  H_1(U \cap V) \xrightarrow{g_*}   H_1(U) \oplus H_1(V) \to  H_1(K) \to  H_0(U \cap  V) \to  H_0(U) \oplus H_0(V) \to  H_0(K) \to  0
        $}

        Klein bottle is path connected,  so

        \[
                    \to  0 \oplus 0 \to  H_2(K) \to  \Z \xrightarrow{g_*}   \Z^2\oplus 0 \to  H_1(K) \to  \Z \to  \Z  \oplus \Z \to  \Z \to  0
        \]

        We can cut of the sequence  because all spaces are path connected.

        \[
                    \to  0 \oplus 0 \to  H_2(K) \to  \Z \xrightarrow{g_*}   \Z^2\oplus 0 \to  H_1(K) \to  0
        \]

        Now, we have to understand what $g_*$ is

        Then

        \[
            \begin{tikzcd}
                U \cap V \arrow[r, "i"] & U \arrow[r, "r"] & \text{figure 8}\\
                H_1(U \cap V) \arrow[r, "i_r"] & H_1(U) \arrow[r, "r_*"] & H_1(\text{figure 8})\\
                \left<c_1 + c_2 + c_3 + c_4 \right> \arrow[r, ""] & \left<c_1+ c_2 +c_3 +c_4 \right> \arrow[r, ""] & \left<r  \circ  c_1 + r  \circ  c_2 + r  \circ  c_3 + r  \circ  c_4 \right>\\
                                      && \left<\overline{a} + \overline{b} + a + \overline{b} \right>
            \end{tikzcd}
        \]
        So we get
        \[
        - \left<a \right> - \left<b \right> + \left<a \right> - \left<b \right> = - 2 \left<b \right>
        .\] 

        So $g_*: \Z \to  \Z^2: 1 \mapsto (0, -2)$, or $z \to  (0, -2 z)$.
        So, the image of $g_* = 0 \times 2 \Z \le  \Z\times \Z$.
        The kernel of $g_*$ is $\{0\}$.
        So $g_*$ is injective.
        And $0 \times  2 \Z$ is the kernel of $h_*$
        \[
                    \to  0 \oplus 0 \to  H_2(K) \to  \Z \xrightarrow{g_*}   \Z^2 \xrightarrow{h_*}   H_1(K) \to  0
        \]
        Now, the zero on the right implies that $h_*$ is surjective.
        so $(H_1(U) \oplus 0) / \operatorname{Ker} h_*$, which is $\frac{\Z^2}{0 \times 2 \Z}$, which is $\Z \oplus \Z_2$.

        Now, on the left.
        As $g_*$ is injective, so $H_2(K) \to  H_1(U \cap V) = \Z$ is the zero map.
        On the other hand, we have zeros on the left, so $H_2(K) \to  H_1(U \cap V) = \Z$ is also injective.
        The only way the zero map can be injective is when the spaces are zero.
        Therefore, $H_2(K) = 0$.

        Higher dimensional: trivial.

        Let $n \ge  3$
        \[
            0 \oplus 0 = H_n(U) \oplus H_n(V) \to  H_n(K) \to H_{n-1}(U \cap V) = 0
        ,\] 
        so $H_n(K) = 0$ for all $n \ge  3$.

        Conclusion:

        \[
            H_0(K) = \Z \quad 
            H_1(K) = \Z \oplus \Z_2 \quad 
            H_2(K) = 0 \quad 
            H_{\ge 3}(K) = 0
        .\] 

        \begin{remark}
            Fact: if $X$ is an $n$-dimensional manifold, which is non orientable, then $H_2$ is always zero.
            If it's orientable, then  $H_2(X) = \Z$.
        \end{remark}

        \begin{remark}
            Fact: The $H_1(X) = \frac{\pi_1(X)}{[\pi_1(X), \pi_1(X)]}$.
            Abelianized version of fundamental group.
            Remember, we did compute the fundamental group of the klein bottle!
            $\pi_1(K) = \left<\alpha, \beta  \mid  \beta \alpha = \alpha^{-1} \beta \right>$.
            Making this group commutative, we have
            \[
                H_1(K) = \left<\alpha, \beta  \mid  \beta\alpha = \alpha^{-1} \beta, \alpha\beta = \beta\alpha \right>
            .\] 
            From this is follows that $\alpha^{-1} \beta = \alpha \beta$, so $\alpha^{-1} = \alpha$, so $\alpha^2 = 1$.
            Therefore, we can rewrite this as
            \[
                H_1(K) = \left<\alpha, \beta  \mid  \alpha\beta = \beta\alpha, \alpha^2 = 1 \right> = \Z_2 \oplus \Z
            .\] 

            Another way is to actually compute commutators.
            \[
                [\alpha, \beta] = \alpha^{-1}\beta^{-1} \alpha \beta = 
                \alpha^{-1} \alpha^{-1} \beta^{-1} \beta^{-1} = \alpha^{-2}
            .\] 
            So, we mod out $\alpha^2$.
        \end{remark}

        \begin{remark}[Exam]
            $H_n(\R P^2)$. The answer should be:
            \begin{itemize}
                \item $H_0 = \Z$ (path connected)
                \item $H_1 = \Z_2$ (abelianized version)
                \item $H_2 = 0$ (orientable)
                \item $H_{n\ge 3} = 0$ 
            \end{itemize}
        \end{remark}
